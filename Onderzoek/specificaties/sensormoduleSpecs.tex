%%%%%%%%%%%%%%%%%%%%%%%%%%%%%%%%%%%%%%%%%
% Thin Sectioned Essay
% LaTeX Template
% Version 1.0 (3/8/13)
%
% This template has been downloaded from:
% http://www.LaTeXTemplates.com
%
% Original Author:
% Nicolas Diaz (nsdiaz@uc.cl) with extensive modifications by:
% Vel (vel@latextemplates.com)
%
% License:
% CC BY-NC-SA 3.0 (http://creativecommons.org/licenses/by-nc-sa/3.0/)
%
%%%%%%%%%%%%%%%%%%%%%%%%%%%%%%%%%%%%%%%%%

%----------------------------------------------------------------------------------------
%	PACKAGES AND OTHER DOCUMENT CONFIGURATIONS
%----------------------------------------------------------------------------------------

\documentclass[a4paper, 11pt]{article} % Font size (can be 10pt, 11pt or 12pt) and paper size (remove a4paper for US letter paper)

\usepackage[protrusion=true,expansion=true]{microtype} % Better typography
\usepackage{graphicx} % Required for including pictures
\usepackage{wrapfig} % Allows in-line images

\usepackage{mathpazo} % Use the Palatino font
\usepackage[T1]{fontenc} % Required for accented characters
%\usepackage[backend=bibtex,style=verbose-trad2]{biblatex}
\usepackage{hyperref}
\usepackage{longtable}
\usepackage{array}
\usepackage{multirow}
\usepackage[utf8]{inputenc}
\usepackage{subcaption}
\usepackage[font=small]{caption}
\usepackage{units}

\linespread{1.05} % Change line spacing here, Palatino benefits from a slight increase by default

\makeatletter
\renewcommand\@biblabel[1]{\textbf{#1.}} % Change the square brackets for each bibliography item from '[1]' to '1.'
\renewcommand{\@listI}{\itemsep=0pt} % Reduce the space between items in the itemize and enumerate environments and the bibliography

\renewcommand{\maketitle}{ % Customize the title - do not edit title and author name here, see the TITLE block below
\begin{flushright} % Right align
{\LARGE\@title} % Increase the font size of the title

\vspace{50pt} % Some vertical space between the title and author name

{\large\@author} % Author name
\\\@date % Date

\vspace{40pt} % Some vertical space between the author block and abstract
\end{flushright}
}

%----------------------------------------------------------------------------------------
%	TITLE
%----------------------------------------------------------------------------------------

\title{\textbf{Specificaties}\\ % Title
Implementaties van koolstofmonoxide sensoren} % Subtitle

\author{\textsc{F. van Beusekom, M. Felida, S. van Bottenburg, J. Grobben, R. Bolding} % Author
\\{\textit{Amsterdam University of Applied Sciences\\ 
HvA}}} % Institution

\date{20 September, 2019} % Date

%----------------------------------------------------------------------------------------
\bibliographystyle{IEEEtran}
\begin{document}
\captionsetup[figure]{labelfont={bf},name={Fig},labelsep=period}
\captionsetup{justification=centering}
\hypersetup{hidelinks=true}
\maketitle % Print the title section

%----------------------------------------------------------------------------------------
%	ABSTRACT AND KEYWORDS
%----------------------------------------------------------------------------------------

%\renewcommand{\abstractname}{Summary} % Uncomment to change the name of the abstract to something else


\vspace{10pt} % Some vertical space between the abstract and first section

%----------------------------------------------------------------------------------------
%	ESSAY BODY
%----------------------------------------------------------------------------------------
\newpage
\section{Onderbouwing}
\subsection{Meeting}
Om de specificaties te bepalen voor de koolmonoxide-meter is er gekeken naar de wettelijke grenswaarden die zijn gegeven in de Arbeidsomstandigheden regeling. Deze zijn $23\,\nicefrac{mg}{m^{3}}$ niet langer dan 8 uur, en $115\,\nicefrac{mg}{m^{3}}$ niet langer dan één kwartier. Omgerekend (en naar beneden afgerond) is dit $18\,ppm$ en $93\,ppm$ respectievelijk. Deze waarden zullen worden genomen als het minimale meetbereik. Om te detecteren of de waarde van $93\,ppm$ langer dan een kwartier wordt overschreven, is een responstijd van één kwartier het absolute maximum. Na het bekijken van meerdere datasheets lijkt een resolutie van minimaal $5\,ppm$ erg goed te realiseerbaar te zijn en dus het streven is hier onder te gaan zitten.
\subsection{Voeding}
De voeding kan op dit moment niet heel veel over worden gezegd, gezien er nog niet bekend is wat voor een batterij er beschikbaar wordt gesteld. Voor het nominale vermogen wordt $1\,mW$ als bovengrens genomen, dit is een getal dat werd genoemd tijdens een gesprek met Marcel van der Horst.

\begin{center}
	\begin{tabular}{ | m{5cm} | m{5cm}| } 
		\hline
		\multicolumn{2}{|c|}{Specificaties voor de implementaties van CO sensoren} \\
		\hline
		Meet bereik a: & <18 - >93 ppm \\
		\hline
		Detectie limiet:  & <18 ppm
		\\ 
		\hline
		Resolutie: & <5 ppm 
		\\ 
		\hline
		Responstijd: & <15 minuten
		\\
		\hline
	\end{tabular}
\end{center}
%----------------------------------------------------------------------------------------

\end{document}
