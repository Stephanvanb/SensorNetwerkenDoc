%%%%%%%%%%%%%%%%%%%%%%%%%%%%%%%%%%%%%%%%%
% Thin Sectioned Essay
% LaTeX Template
% Version 1.0 (3/8/13)
%
% This template has been downloaded from:
% http://www.LaTeXTemplates.com
%
% Original Author:
% Nicolas Diaz (nsdiaz@uc.cl) with extensive modifications by:
% Vel (vel@latextemplates.com)
%
% License:
% CC BY-NC-SA 3.0 (http://creativecommons.org/licenses/by-nc-sa/3.0/)
%
%%%%%%%%%%%%%%%%%%%%%%%%%%%%%%%%%%%%%%%%%

%----------------------------------------------------------------------------------------
%	PACKAGES AND OTHER DOCUMENT CONFIGURATIONS
%----------------------------------------------------------------------------------------

\documentclass[a4paper, 11pt]{article} % Font size (can be 10pt, 11pt or 12pt) and paper size (remove a4paper for US letter paper)

\usepackage[protrusion=true,expansion=true]{microtype} % Better typography
\usepackage{graphicx} % Required for including pictures
\usepackage{wrapfig} % Allows in-line images

\usepackage{mathpazo} % Use the Palatino font
\usepackage[T1]{fontenc} % Required for accented characters
%\usepackage[backend=bibtex,style=verbose-trad2]{biblatex}
\usepackage{hyperref}
\usepackage{longtable}
\usepackage{array}
\usepackage{multirow}
\usepackage[utf8]{inputenc}
\usepackage{subcaption}
\usepackage[font=small]{caption}

\linespread{1.05} % Change line spacing here, Palatino benefits from a slight increase by default

\makeatletter
\renewcommand\@biblabel[1]{\textbf{#1.}} % Change the square brackets for each bibliography item from '[1]' to '1.'
\renewcommand{\@listI}{\itemsep=0pt} % Reduce the space between items in the itemize and enumerate environments and the bibliography

\renewcommand{\maketitle}{ % Customize the title - do not edit title and author name here, see the TITLE block below
\begin{flushright} % Right align
{\LARGE\@title} % Increase the font size of the title

\vspace{50pt} % Some vertical space between the title and author name

{\large\@author} % Author name
\\\@date % Date

\vspace{40pt} % Some vertical space between the author block and abstract
\end{flushright}
}

%----------------------------------------------------------------------------------------
%	TITLE
%----------------------------------------------------------------------------------------

\title{\textbf{Specificaties}\\ % Title
Implementaties van koolstofmonoxide sensoren} % Subtitle

\author{\textsc{F. van Beusekom, M. Felida, S. van Bottenburg, J. Grobben, R. Bolding} % Author
\\{\textit{Amsterdam University of Applied Sciences\\ 
HvA\\
Sensor Netwerken: groep 5}}} % Institution

\date{20 september, 2019} % Date

%----------------------------------------------------------------------------------------
\bibliographystyle{IEEEtran}
\begin{document}
\captionsetup[figure]{labelfont={bf},name={Fig},labelsep=period}
\captionsetup{justification=centering}
\hypersetup{hidelinks=true}
\maketitle % Print the title section

%----------------------------------------------------------------------------------------
%	ABSTRACT AND KEYWORDS
%----------------------------------------------------------------------------------------

%\renewcommand{\abstractname}{Summary} % Uncomment to change the name of the abstract to something else


\vspace{10pt} % Some vertical space between the abstract and first section

%----------------------------------------------------------------------------------------
%	ESSAY BODY
%----------------------------------------------------------------------------------------
\newpage
\section{Inleiding}
In dit verslag worden de specificaties besproken van de te ontwikkelen sensor modules voor het project "Sensor Netwerken". Bij dit vak is het de bedoeling dat er low-power sensor modules worden ontwikkeld die de luchtkwaliteit in het HvA gebouw van de faculteit techniek meten en als netwerk informatie uitwisselen met elkaar. Binnen groep 5 is er voor gekozen om koolstofmonoxide te meten (hierna naar gerefereerd als CO). Uiteindelijk moet iedereen van deze groep zijn eigen sensor module ontwikkelen. Aangezien het HvA gebouw van de faculteit techniek te vergelijken is met de meeste professionele werkomgeving, wordt er voor dit onderzoek gekeken naar de gebruikelijke CO concentraties in gebouwen en hun uitschieters. Op basis daarvan worden de specificaties opgesteld voor de sensor modules.

\section{Algemene koolstofmonoxide concentraties}
Het NIVM heeft tussen april 2007 tot en met januari 2008 metingen gedaan in 1028 huishoudens. In 169 van de 1028 huishoudens werden werden er concentraties boven het detectie limiet van 1 \textit{parts per billion} (hierna naar gerefereerd als \textit{ppm}) gevonden. In 8 huishoudens werden er waardes van tussen de 25 en 75 \textit{ppm}, dit waren de meest extreme waardes. in 10 andere huishoudens werden er waardes aangetroffen van tussen de 10 en 25 \textit{ppm} en in de rest van de huishoudens (verreweg de meesten) werden er waardes aangetroffen tussen de 1 en 9 \textit{ppm} [1]. Over een werkdag van 8 uur wordt een tijdgewogen gemiddelde concentratie van 25 \textit{ppm} als grenswaarde aangenomen, over 15 minuten is dit 150 \textit{ppm}. De advies waarden van de Wereldgezondheidsorganisatie (Engels: World Health Organization, WHO) zijn 10 \textit{ppm} als tijdgewogen gemiddelde over 8 uurl en 25 \textit{ppm} over 1 uur [2].
\subsection{Koolstofmonoxide concentraties in de HvA Leeuwenburg}
Om meer te weten te komen over de huidige manieren van meten van de koolstofmonoxide concentraties in het gebouw van de HvA faculteit techniek, is er contact gezocht het gebouwbeheer. Uit een klein gesprek is gebleken dat er momenteel alleen in de garage CO meters hangen, die wanneer er een bepaalde grenswaarde wordt overschreden (momenteel nog onbekend voor ons welke grenswaarde dat is) alarm slaan waarna vervolgens iedereen de garage moet verlaten.

\section{Luchtkwaliteit meten}
Zoals eerder werd besproken worden de te ontwikkelen sensor modules gebruikt om de luchtkwaliteit te meten. Hierbij wordt dus niet alleen bedoeld om een detector te maken die afgaat na het overschrijden van een bepaalde waarde, maar is het een doel om ook te kunnen meten wanneer je welke concentratie CO in de lucht hebt zitten en hoe ver deze concentratie ligt van de grenswaardes. Binnen groep 5 is afgesproken om de aanname te doen dat de minimale grenswaarde in delen van 1/5e gemeten moet kunnen worden. Aangezien onze sensor modules de hele dag door de CO concentraties in het gebouw moeten meten, gaan we uit van de grenswaarden voor een werk dag (8 uur lang). In dit geval komt de minimale grenswaarde voort uit de Wereldgezondheidsorganisatie (Engels: World Health Organization, WHO) en bedraagt deze 10 \textit{ppm} als tijdgewogen gemiddelde over de hele werkdag. Dit betekend dat wij de CO concentraties willen met een gevoeligheid van 2 \textit{ppm}. Het detectie limiet moet ook 1/5e zijn van de minimale grenswaarde. Dit bedraagt 2 \textit{ppm}.

\section{Specificaties}
\begin{center}
	\begin{tabular}{ | m{5cm} | m{5cm}| } 
		\hline
		\multicolumn{2}{|c|}{Specificaties voor de implementaties van CO sensoren} \\
		\hline
		Meet bereik a: & 0 - 1000 \textit{ppm} \\
		\hline
		detectie limiet:  & 1 \textit{ppm}
		\\ 
		\hline
		detectie resolutie: & 1 \textit{ppm} 
		\\ 
		\hline
		response tijd: & < 1 minuut
		\\ 
		\hline
		Voed spanning: & min: 2,7V max: 3,3V
		\\ 
		\hline
		Maximaal vermogen: & 1 mW
		\\
		\hline
		Output gevoeligheid: & 1mV/\textit{ppm}
		\\
		\hline
	\end{tabular}
\end{center}
%----------------------------------------------------------------------------------------
\newpage
\begin{thebibliography}{9}
	\bibitem{NIVM huurwoningen}
	M. van Bruggen, J.T.M. Gram, E.L. Boels, L. Ruhaak, M. Mooij,
	NIVM,
	2009 [Bekeken in september 2019],
	[Rapport],
	"Koolmonoxide in huurwoningen in de Randstad",
	Beschikbaar: \url{https://www.rivm.nl/bibliotheek/rapporten/609300009.pdf}
	
	\bibitem{Blootstelling aan CO}
	M. Mooij,
	NIVM,
	2008 [Bekeken in september 2019],
	[Rapport],
	"Chronische blootstelling aan koolmonoxide, tabel 2.2",
	Beschikbaar: \url{https://www.rivm.nl/bibliotheek/rapporten/609300005.pdf}
	
\end{thebibliography}

\end{document}
