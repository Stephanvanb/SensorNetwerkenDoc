\documentclass{article}
\usepackage[utf8]{inputenc}

\title{Effecten CO}
\date{September 2019}

\usepackage{natbib}
\usepackage{graphicx}

\begin{document}

\maketitle
\newpage
\section{Koolstofmonoxide}
Koolstofmonoxide (CO) is een verbinding tussen koolstof en zuurstof. Het is een kleur en geurloos gas. CO komt vrij bij een onvolledige verbranding. Dit kan erg gevaarlijk zijn voor de gezondheid omdat het gas je ongemerkt vergiftigt. Het word dan ook wel eens een "sluipmoordenaar" genoemd. 

\subsection{Ontstaan van CO}
Bij de verbranding van koolstofverbindingen ontstaat er normaal CO2. Als de verbranding onvolledig is, door bijvoorbeeld een tekort aan zuurstof, ontstaat er CO. Een veel voorkomende oorzaak van CO productie zijn niet goed afgestelde of niet goed onderhouden gas kachels en cv-ketels.

\subsection{Effecten op gezondheid}
CO is erg gevaarlijk voor de mens omdat het geur en kleurloos is. Bij inademing van CO word de zuurstofopname verstoord. In plaats van zuurstof, bind de CO zich aan de rode bloedcellen. Omdat CO zich sterker hecht aan de rode bloedcellen dan zuurstof, word er minder zuurstof door het lichaam vervoerd. 

De effecten van blootstelling aan CO zijn het gevolg van een zuurstoftekort in het bloed. Hoe meer CO er word opgenomen hoe minder zuurstof er kan worden opgenomen. 


\begin{table}[h]
\begin{tabular}{|l|l|}
\hline
\textbf{Concentratie CO (ppm)} & \textbf{Effecten van CO (tijd gebonden gemiddelde)}                                                                                   \\ \hline
6                              & Maximaal toelaatbare concentratie                                                                                                     \\ \hline
150                            & Lichte hoofdpijn na 1,5 uur.                                                                                                          \\ \hline
200                            & Lichte hoofdpijn, vermoeidheid, misselijkheid na 2-3 uur.                                                                             \\ \hline
400                            & Frontale hoofdpijn binnen 1-2 uur levensbedreigend na 3 uur.                                                                          \\ \hline
800                            & \begin{tabular}[c]{@{}l@{}}Duizeligheid, misselijkheid en stuiptrekkingen binnen 45 minuten. \\ Overlijden na 2-3 uur.\end{tabular}   \\ \hline
1600                           & \begin{tabular}[c]{@{}l@{}}Hoofdpijn, duizeligheid en misselijk binnen 20 minuten. \\ Overlijden binnen 1 uur.\end{tabular}           \\ \hline
3200                           & \begin{tabular}[c]{@{}l@{}}Hoofdpijn, duizeligheid en misselijk binnen 5-10 minuten. \\ Overlijden binnen 25-30 minuten.\end{tabular} \\ \hline
6400                           & \begin{tabular}[c]{@{}l@{}}Hoofdpijn, duizeligheid en misselijk binnen 1-2 minuten. \\ Overlijden binnen 10-15 minuten.\end{tabular}  \\ \hline
12800                          & Overlijden binnen 1-3 minuten.                                                                                                        \\ \hline
\end{tabular}
\end{table}
\end{document}
